\ifthenelse{\value{cols}=1}{
    \setlength{\columnsep}{20pt} % Spacing tra le colonne
    \begin{multicols}{2}[\section{Results and Conclusion}]
}{
    \section{Conclusion}
}
    In this chapter, we report some final observations about our agent, including results obtained on various challenge maps after extensive testing under tournament conditions, along with a potential direction for future improvements.

    \subsection{Results}
        We conducted a series of tests on selected maps from both challenges, chosen to evaluate the key components of the single-agent and multi-agent systems. Multiple agent instances were executed in parallel to simulate competitive real-world conditions.
        \medskip\\
        For the single-agent tests, we evaluated four maps from the first challenge using two modes: no planning (NP) and planning with 50\% probability (5P). In the planning tests, we enabled the \textit{-w} parameter to block new option generation during action execution. Previous tests conducted without this parameter, allowing the \textit{option generation} process to run and potentially override the current action, generally yielded slightly worse results.
        \medskip\\
        As shown in Table \ref{tab:table_1}, comparing cumulative scores across the same map with and without planning, the use of planning degraded performance. This effect was especially pronounced in smaller maps with narrow corridors or limited spawning tiles.
        \medskip\\
        For the multi-agent tests, we evaluated four maps from the second challenge, along with the hallway map. Results are reported in Table \ref{tab:table_2}. In these cases, no major anomalies were observed: the agents operated effectively and were able to exchange parcels when required.
    
    \subsection{Possible Improvements}
        We observed that the agent does not actively perform sensing at will, instead, the server automatically sends updated information. While this behavior is appropriate in most situations, the server only sends updates when changes occur. As a result, internal predictions are necessary to maintain accurate beliefs. One case involves parcel expiration: the server does not notify agents when a parcel expires, so the agent must infer this internally. Another case concerns other moving agents: if an agent observes another agent in motion, the server promptly sends updates. However, if the observing agent moves and loses visual contact, no further updates are sent. The agent must then infer that the unseen agent has likely moved away and should be removed from memory.
        \medskip\\
        Both systems were implemented using a time-window approach (Section \ref{memoryRevision}). This method works effectively for parcels. However, in the case of agents, a specific issue arises: if another agent remains stationary, the server sends only a single update upon entering the sensing range. With no subsequent updates, the time window eventually expires, and the agent is erroneously removed from memory. While not critical, since agents typically remain in motion, this limitation represents a possible area for future improvement.
\ifthenelse{\value{cols}=1}{
    \end{multicols}
}{
    % Pass
}

\newpage

\begin{table}[H]
    \centering
    \begin{tabular}{|c|c|c|c|c|c|c|}
        \hline
        Map & Teams & First & Second & Third & Total Score & Percentage First \\
        \hline
        25c1\_1 (NP)  & 10   & 230   & 210   & 190  & 1600  & 14.4\% \\
        25c1\_1 (5P)  & 10   & 250   & 160   & 140  & 1270  & 19.7\% \\
        25c1\_2 (NP)  & 10  & 1134  & 1094  & 1047  & 8164  & 13.9\% \\
        25c1\_2 (5P)  & 10   & 534   & 483   & 455  & 3452  & 15.5\% \\
        25c1\_6 (NP)   & 5   & 363   & 196   & 167  & 1005  & 36.1\% \\
        25c1\_6 (5P)   & 5   & 189   & 179   & 175   & 739  & 25.6\% \\
        25c1\_8 (NP)   & 5   & 833   & 694   & 680  & 3539  & 23.5\% \\
        25c1\_8 (5P)   & 5   & 258   & 214   & 161   & 901  & 28.6\% \\
        \hline
    \end{tabular}
    \caption{Testing results for the single-agent. In order, we report the map name (presence of planning), the number of competing teams, first place score, second place score, third place score, total cumulative score of all teams and percentage of the first place score with respect to the total score}
    \label{tab:table_1}
\end{table}

\begin{table}[H]
    \centering
    \begin{tabular}{|c|c|c|c|c|c|c|}
        \hline
        Map & Teams & First & Second & Third & Total Score & Percentage First \\
        \hline
        25c2\_3         & 5  & 2172   & 1811  & 1721  & 8462  & 25.7\% \\
        25c2\_5         & 3  & 1468   & 1432  & 1370  & 4270  & 34.4\% \\
        25c2\_6         & 5  & 1941   & 1601  & 1579  & 7947  & 24.4\% \\
        25c2\_7         & 5  & 1805   & 1738  & 1555  & 7859  & 23.0\% \\
        25c2\_hallway   & 1  & 1471   & -     & -     & 1471   & 100\% \\
        \hline
    \end{tabular}
    \caption{Testing results for the multi-agent. In order, we report the map name, the number of competing teams, first place score, second place score, third place score, total cumulative score of all teams and percentage of the first place score with respect to the total score}
    \label{tab:table_2}
\end{table}
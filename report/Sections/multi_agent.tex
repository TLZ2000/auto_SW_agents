\ifthenelse{\value{cols}=1}{
    \setlength{\columnsep}{20pt} % Spacing tra le colonne
    \begin{multicols}{2}[\section{Multi Agent}]
}{
    \section{Multi Agent}
}
    This second part concerns the development of the multi-agent system, specifically the addition of a second agent (\textit{"pal"}) capable of operating in the \textit{Deliveroo.js} environment. The two agents must be able to communicate, update their internal beliefs with shared information, coordinate and negotiate common solutions (e.g., deciding who should pick up a specific parcel or collaborating to achieve goals).
    \medskip\\ 
    As previously stated, the multi-agent script is the same as the single-agent one, with the multi-agent mode enabled through a parameter when launching the agent's script.

    \subsection{Additions to the Single Agent}
        Compared to the standard single-agent components, several key elements were added to ensure multi-agent compatibility. This design allows the multi-agent system to function even in the absence of the \textit{pal}, providing resilience in cases where the second agent unexpectedly disconnects.
        \medskip\\
        As in the single-agent case, the multi-agent implementation includes two specific multi-agent tokens with associated ids. This enables both agents (i.e., \textit{agent1} and \textit{agent2}) to communicate. The following describes the main additions relative to the single-agent implementation.


        \subsubsection{Pal Memory}
            The first step is to define an additional memory structure to store all information related to the \textit{pal}, specifically: its id, position, current intention, and the timestamp of the last received update. The \textit{pal}'s id is set at the beginning of the agent script, as both ids and tokens are hardcoded. The remaining information is set and updated through message-based communication.
            \medskip\\
            Another modified component is the \textit{reviseMemory} function. If the \textit{pal} fails to communicate within a defined time window, it is considered disconnected, and its memory data is cleared. This enables the agent, despite continuing to send non-blocking \textit{say} messages to the now-nonexistent \textit{pal}, to operate as a single-agent under the assumption that the \textit{pal} is absent and carrying no parcels.

            
        \subsubsection{Communication}            
            The most significant difference compared to the single-agent implementation is the presence of a communication component enabling agents to send and receive messages. Four message types are defined, one of which is described in detail in Section \ref{shareRequest}. The remaining three message types allow agents to synchronize beliefs and operate accordingly. Following a rapid-option-generation strategy, and after multiple unsuccessful attempts with structured message sharing, a "\textit{UDP-like}" message-sharing strategy was adopted for these three types: if the \textit{pal} is listening, it updates its beliefs, otherwise, no disruption occurs. Similarly, for message reception: if the \textit{pal} is active and sends updates, the agent updates its beliefs, otherwise, it continues operating with the current belief state without issue.
            \medskip\\
            A brief description of the three "\textit{UDP-like}" message types follows:
            \begin{itemize}
                \item "\textbf{\textit{MSG\_positionUpdate}}": in every \textit{onYou} callback, this message shares the agent's current position with the \textit{pal}. Upon receiving it, the agent updates its belief about the \textit{pal}'s position, sets the new \textit{last update} timestamp, updates the \textit{pal}'s entry in the \textit{agent memory}, and modifies its internal \textit{time map}, enabling spatial differentiation in exploration relative to the \textit{pal};                    
                \item "\textbf{\textit{MSG\_memoryShare}}": after each \textit{memory revision}, the agent sends the updated version of its memory, specifically the \textit{parcel} and \textit{agent} memories, the parcels it is currently carrying, and its position. Upon receiving this message, the agent stores the \textit{pal}'s carried parcels in a dedicated variable, removes from its \textit{parcel memory} any parcels believed to be carried by the \textit{pal} (re-adding them later if still present), and updates its \textit{parcel} and \textit{agent} memories with the received data. Finally, it invokes the handler for the \textit{MSG\_positionUpdate} message to update the \textit{pal}'s position.
                \item "\textbf{\textit{MSG\_currentIntention}}": when pushing a new \textit{next intention}, the agent also sends the intention name to the \textit{pal}. Upon receiving this message, the agent updates its belief regarding the \textit{pal}'s current intention.
            \end{itemize}
            
            \subsubsection{Options and Filtering}
                The presence of a collaborative \textit{pal} enables improvements in the option generation and filtering procedures. The first enhancement concerns the generation of the best "\textit{go\_pick\_up}" option, particularly in reward computation. When processing individual parcels, the agent computes a reward for each parcel for both itself and the \textit{pal}, using internal information without requiring communication. If the \textit{pal} is absent from the agent's memory, the \textit{pal}'s reward defaults to zero. This additional reward assessment allows for refined filtering: if the agent's reward exceeds the \textit{pal}'s reward, or the rewards are equal but the agent's distance to the parcel is shorter, the agent generates an associated option. Additionally, if the \textit{pal}'s current intention is "\textit{go\_deliver}", the agent assumes that the \textit{pal} will ignore available parcels and will always generate an option.
                \medskip\\
                The second and most significant improvement is the introduction of a new option type: "\textit{share\_parcels}". This option addresses "\textit{corridor-like}" scenarios, where one agent can pick up parcels but cannot deliver them, while the other can deliver parcels but not pick them up. This option is pushed when no \textit{best option} is identified (i.e., no "\textit{go\_pick\_up}" or "\textit{go\_deliver}" options), the agent is carrying parcels, and there is a reachable path to the \textit{pal}. To distinguish between temporarily obstructed delivery paths and persistent blockages, a dedicated counter is introduced. This ensures that the agent commits to a "\textit{share\_parcels}" intention only when the path is really obstructed, and not when the \textit{pal} momentarily traverses a single blocking cell.
            
            \subsubsection{Plans}\label{shareRequest}
                As previously stated, to handle \textit{corridor-like} situations, a new "\textit{share\_parcels}" intention was introduced. It is triggered by the agent carrying parcels but being unable to deliver them due to the \textit{pal} blocking the delivery path. To prevent the generation of new options from disrupting the cooperation phase, a dedicated flag was added. This flag is activated during the execution of the following plans and blocks the push of new intentions while the collaboration phase is ongoing.

                \paragraph{ShareParcels Plan}
                    To handle the new intention, a dedicated plan \textit{ShareParcels} was introduced. The first step in this plan is to send a \textit{MSG\_shareRequest} message containing the agent's current position, using the \textit{emitAsk} utility to await the \textit{pal}'s response. The \textit{pal}, or the server, may respond in one of the following ways:
                    \begin{itemize}
                        \item \textbf{"timeout" or null}: in the first case, the \textit{pal} fails to respond within the expected time window, resulting in a "timeout" from the server. The second case occurs in the single-agent setting, where the \textit{emitAsk} wrapper automatically returns \textit{null}. In both cases, the intention fails immediately;
                        \item \textbf{"false"}: the \textit{pal} explicitly refuses the request, causing the intention to fail immediately;
                        \item \textbf{"you\_move"}: the agent is obstructing the \textit{pal} (handling the "\textit{share\_parcels}" intention requires a path of at least four positions between the agents). The agent must move to create sufficient space. If no valid path is available, the intention fails. Otherwise, once the agent has moved and sufficient space exists, it sends another \textit{MSG\_shareRequest} message;
                        \item \textbf{"true"}: the \textit{pal} accepts the request and sends the coordinates required for the parcel exchange, in particular a \textit{wait position} (where the \textit{pal} will wait), an \textit{exchange position} (where the agent will drop parcels), and an adjacent \textit{support position} (where the agent will move after dropping the parcels and await pickup by the \textit{pal}).
                    \end{itemize}                   
                    Upon receiving a \textit{"true"} message, the agent proceeds with the intention. First, it moves to the \textit{exchange position} and waits for the \textit{pal} to arrive at the \textit{wait position}. If the \textit{pal} changes its intention before arrival, this implies a failure in the parcel recovery plan (e.g., due to collision or path obstruction), and the agent's intention also fails. Otherwise, the agent drops its parcels at the exchange position, moves to the \textit{support position}, and waits for the \textit{pal} to complete the pickup. The intention is then considered successfully completed.

                
                \paragraph{Handling the MSG\_shareRequest Message Type}
                    When the agent receives a \textit{MSG\_shareRequest} message, it first updates the \textit{pal}'s position in memory and computes a path. If the agent is carrying parcels, the path includes also the nearest delivery cell before reaching the \textit{pal}, otherwise, it is a direct path to the \textit{pal}. If no valid path exists, the agent responds with \textit{"false"} and the intention is rejected. Otherwise, the path is analyzed to extract the three required positions: the midpoint becomes the \textit{wait position}, the next cell (from the agent's perspective) becomes the \textit{exchange position}, and the following cell becomes the \textit{support position}. If these positions are valid, the agent pushes a "\textit{recover\_shared\_parcels}" option to commit to the exchange and responds with \textit{"true"}, including the computed positions.
                    \medskip\\
                    Due to the structure of this mechanism, the path must be at least four positions long. If this condition is not met, the agent checks whether it can reach a delivery position and whether the resulting path allows for at least four free positions for the exchange. If so, it commits to the movement and then retries the process. Otherwise, it responds with \textit{"you\_move"}, indicating that the \textit{pal} must create space, and includes the number of required free cells.
                    
                \paragraph{RecoverSharedParcels Plan}
                    To handle the "\textit{recover\_shared\_parcels}" intention, a new plan \textit{RecoverSharedParcels} has been introduced. Unlike the previous case, the agent already knows the three exchange positions. If a delivery is required (as determined by the message handler), it is completed first. The agent then moves to the \textit{wait position} and waits for the \textit{pal} to reach the \textit{exchange position}, drop the parcels, and move to the \textit{support position}. As in the \textit{ShareParcels} plan, if the \textit{pal} changes its current intention, indicating a failed share intention, the agent's intention also fails. Otherwise, the agent proceeds to the \textit{exchange position}, picks up the parcels, and successfully completes the intention.

\ifthenelse{\value{cols}=1}{
    \end{multicols}
}{
    % Pass
}
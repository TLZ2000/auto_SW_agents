\ifthenelse{\value{cols}=1}{
    \setlength{\columnsep}{20pt} % Spacing tra le colonne
    \begin{multicols}{2}[\section{Introduction}]
}{
    \section{Introduction}
}
    This document contains the final report for the Autonomous Software Agents project, in which one or more agents are required to play the Deliveroo.js game. Agents developed for this project must implement the BDI (\textit{Beliefs, Desires, Intentions}) architecture (Fig. \ref{BDI_architecture}), where each agent senses the environment to form internal beliefs, generates a set of possible intentions, and commits to one or more of them via a plan-based system, while continuously revising both intentions and beliefs.
    \captionedImage{46}{BDI_architecture}{BDI architecture diagram used during the agent development}
    The project is composed of a single-agent part and a multi-agent part. The first part involves a single agent implementing the basic functions necessary for correct operation. Specifically, the agent must represent and manage beliefs derived from sensing data, activate intentions, act on the environment, and use predefined plans to achieve its intentions. This occurs while continuously revising beliefs and intentions, allowing the agent to stop, hold, or invalidate the currently running intention. This enables the agent to respond appropriately in a rapidly evolving environment. Once these functions are implemented, the agent should interact with an automated planning utility to obtain the sequence of actions to perform
    \medskip\\      
    The second part introduces a second agent capable of cooperating with the first one, and vice versa, to achieve the goal. Specifically, the two agents must be able to communicate, exchange beliefs, coordinate, and negotiate possible solutions to reach the goal, which may be unattainable by a single agent.
    \medskip\\
    Furthermore, both agents must operate effectively in different scenarios involving other competitive agents and rapidly evolving environments.
\ifthenelse{\value{cols}=1}{
    \end{multicols}
}{
    % Pass
}